\begin{frame}[fragile]
\frametitle{Assertions}
\begin{python}
def stddev(values):
    '''
    S = stddev(values)

    Compute standard deviation
    '''
    assert len(values) > 0, 'stddev: got empty list.'
    ...
\end{python}
\end{frame}

\begin{frame}[fragile]
\frametitle{Assertions}

\begin{python}
def stddev(values):
    '''
    S = stddev(values)

    Compute standard deviation
    '''
    if len(values) <= 0:
        raise AssertionError(
            'stddev: got empty list.')
    ...
\end{python}

\end{frame}

\begin{frame}[fragile]
\frametitle{Preconditions}

\begin{quote}
In computer programming, a precondition is a condition or predicate that must always be true just prior to the execution of some section of code.
\end{quote}

\begin{flushright}
(Wikipedia)
\end{flushright}

\end{frame}

\begin{frame}[fragile]
\frametitle{Assertions Are Not Error Handling!}

\begin{itemize}
\end{itemize}
\end{frame}

\begin{frame}[fragile]
\frametitle{Programming by Contract}
\begin{enumerate}
\end{enumerate}
\end{frame}

\begin{frame}[fragile]

\begin{block}{Pre-condition}
What must be true before calling a function.
\end{block}

\begin{block}{Post-condition}
What is true after calling a function.
\end{block}

\end{frame}

\end{frame}


\begin{frame}[fragile]
\frametitle{Unit Testing}

\begin{python}
def test_stddev_const():
    assert stddev([1]*100) < 1e-3

def test_stddev_positive():
    assert stddev(range(20)) > 0.
\end{python}

\end{frame}

\end{enumerate}
\end{frame}

\begin{frame}[fragile]
\frametitle{Practical Session: some preliminaries}
\begin{block}{statistics.py}
\begin{python}
def stddev(xs):
    . . .
\end{python}
\end{block}

\begin{block}{test\_statistics.py}

\begin{python}

def test_stddev_const():
    assert stddev([1]*100) < 1e-3

def test_stddev_positive():
    assert stddev(range(20)) > 0.
\end{python}
\end{block}

\end{frame}
\begin{frame}[fragile]
\frametitle{Practical Session: some preliminaries}
\begin{block}{statistics.py}
\begin{python}
def stddev(xs):
    . . .
\end{python}
\end{block}

\begin{block}{test\_statistics.py}

\begin{python}
import statistics
def test_stddev_const():
    assert statistics.stddev([1]*100) < 1e-3

def test_stddev_positive():
    assert statistics.stddev(range(20)) > 0.
\end{python}
\end{block}

\end{frame}

\begin{frame}[fragile]
\frametitle{Practical: Python III \& Unit testing}

\begin{enumerate}
\item You can either start from scratch or check the files I give you\\
    (or any combination of both).
\item Goal is to write code to do a simple task \& test it.
\end{enumerate}

\end{frame}


\begin{frame}[fragile]
\frametitle{Goals}
\begin{enumerate}
\item Copy files from scratch.
\item There is a data file (\texttt{data.txt})
\item See the code in \texttt{main.py}, which loads it.
\item Write a function \emph{average} in a file called \texttt{robust.py}, which computes the average of a sequence of numbers, whilst ignoring the maximum and minimum.
\item Write tests for \texttt{robust.average}.
\item (If you have the time, you can look at \texttt{plots.py})
\end{enumerate}
\end{frame}

\end{document}
